\documentclass[a4paper 14pt]{article}
\usepackage[utf8]{inputenc}
%\usepackage[pdftex]{graphicx}
\usepackage[margin=1in]{geometry}
\usepackage{natbib}
\usepackage{epsfig}
\usepackage{amsmath}
\usepackage{amsfonts}
\usepackage{float}
\usepackage{rotating} 
\usepackage{caption}
\usepackage{subfig}
\usepackage{booktabs}
\usepackage{adjustbox}
\usepackage[table]{xcolor}
\usepackage{tabularx}
\usepackage{caption}
\usepackage{enumerate}
\usepackage{enumitem}
\captionsetup{font=footnotesize}
\newcommand{\ra}[1]{\renewcommand{\arraystretch}{#1}}
\textheight 9.0 in
\textwidth 6.5 in
\topmargin -0.5 in
\oddsidemargin 0.0in
\renewcommand{\topfraction}{1}
\renewcommand{\bottomfraction}{1}
\renewcommand{\textfraction}{0}
\renewcommand{\floatpagefraction}{0.90}
\definecolor{TableEven}{rgb}{0.8000,0.9216,0.9490}
\usepackage{makecell}
%\usepackage{fourier} 
\numberwithin{equation}{section}
\usepackage{array}
\usepackage{titlesec}
\usepackage{sectsty}
\sectionfont{\centering}
\setcounter{secnumdepth}{4}
\usepackage{natbib}
\usepackage{siunitx}
\usepackage[toc,page]{appendix}

 
\renewcommand{\baselinestretch} {2.0}
\makeatletter
\setcounter{page}{1}
\def\doublespace{\def\baselinestretch{1}\@normalsize}
\def\enddoublespace{}
\title{\bf 
}   
% \footnotemark}
\author{}
\date{}
\@addtoreset{equation}{section}
\renewcommand{\sp}{\vspace{0.2 in}}
\renewcommand{\theequation} {\arabic{section}.\arabic{equation}}
%\renewcommand{\thefigure}{\arabic{section}.\arabic{figure}}
\renewcommand{\thefootnote}{\fnsymbol{footnote}}
\newtheorem{theorem}{Theorem}
\newtheorem{lemma}{Lemma}[section]
\newtheorem{remark}{Remark}[section]
\newtheorem{corollary}{Corollary}[section]
\newtheorem{exam}{Example}[section]
\newtheorem{proposition}{Proposition}[section]

\newcommand{\Bigskip}{\vspace{0.3 in}}

\usepackage{xspace}
\newcommand{\m}{\textnormal{\sffamily m}\xspace}
\newcommand{\cm}{\textnormal{\sffamily cm}\xspace}
\newcommand{\g}{\textnormal{\sffamily g}\xspace}
\newcommand{\kg}{\textnormal{\sffamily kg}\xspace}


\makeatletter
\let\latex@xfloat=\@xfloat
\def\@xfloat #1[#2]{%
  \latex@xfloat #1[#2]%
  \def\baselinestretch{1}
  \@normalsize\normalsize
  \normalsize
}
\makeatother

\newcommand\longitude[1]{\directlua{ longitude ( \luastring{#1} ) }}


\usepackage{lineno}
\linenumbers


 
\begin{document}
\title{An Analysis of the North Sea International Bottom Trawl Survey}
%Uncertainty estimation of abundance indices for North Sea International Bottom Trawl Survey Data
\maketitle


\begin{abstract}


\fontsize{11.3}{13.5}\selectfont
The North Sea International Bottom Trawl Survey (IBTS) was started by the International Centre for the  Exploration of the Sea (ICES) in 1990. Seven research vessels using standardized fishing methods participates in the survey. The survey with these vessels, which allows fishing also on rough ground provides information on seasonal distribution of stocks and abundance,  which forms the basis for stock assessments for many fish stocks in the North Sea. Point estimates of abundance at age from IBTS are provided without any estimates of precision,  and these should not be published or used unless they are accompanied by estimates of uncertainty. Variance estimates of parameters relating to stock size can have a profound effect on the formulation of management policies, and can be used in determining adequate levels of sampling effort in terms of number of days at sea, number of primary sampling units and number of samples for age determination. The point estimates of abundance at age from IBTS are currently obtained by using an age-length key (ALK) method that assumes age compositions are the same over relatively large areas: that assumption is not credible and will give bias results. We developed ALK estimators that account for spatial variation in age-length compositions and provide estimates of uncertainty of abundance at age in fish stocks in the North Sea.  \\


%The North Sea International Bottom Trawl Survey (IBTS) was started by the International Centre for the  Exploration of the Sea (ICES) in 1990. Seven research vessels using standardized fishing methods participates in the survey. The survey with these vessels, which allows fishing also on rough ground provides information on seasonal distribution of stocks and abundance,  which forms the basis for stock assessments for many fish stocks in the North Sea. Point estimates of abundance at age from IBTS are provided without any estimates of precision,  and these should not be published or used unless they are accompanied by estimates of uncertainty. Variance estimates of parameters relating to stock size can have a profound effect on the formulation of management policies. These variance estimates can further be used to determine adequate levels of sampling effort in terms of number of days at sea, number of primary sampling units and number of samples for age determination. The point estimates of abundance at age from IBTS are currently obtained by using an age-length key (ALK) method that assumes age compositions are the same over relatively large areas: that assumption is not credible and will give bias results. We developed ALK estimators that account for spatial variation in age-length compositions and provide estimates of uncertainty of abundance at age in fish stocks in the North Sea.  \\

%
%Estimates of measurement of variance of the parameter estimates relating to stock can have a profound effect management policies.  \\
%
%
%
%%can have a profound effect on mana
%
%Estimates of parameters relating to stock size are of little value unless they are accompanied by estimates of precision. Ignoring estimates of measurement \\
%
%The effect on management \\
%
%% and these cannot be ignored when formulating management policies.  Lack of \\
%
%
% Furthermore, these point estimates are obtained using an age-length key (ALK) method that assumes age compositions are the same over relatively large areas: that assumption is not credible and will give bias results. \\
%
%We develop \\
%
%Further these point estimates are obtained using an age-length-key (ALK) that is applied to populations where the age composition differs from that of the population from which the ALK was drawn. \\
%
%An age-length key (ALK) method \\
%
% Point estimates of abundance at age, based on  age-length keys (ALKs) over a relatively large area are provided without any precision \\
%
%Estimates of abundance indices based on age-length keys (ALK) are provided without any precision . \\
% 
% importance of estimates of precision.....on average  x number of days at sea, with x amount of trawl hauls. No justifcation for  so many days or trawl hauls ....\\
% 
%  We present a model-based ALK estimator, and a stratified design-based ALK estimator for estimating abundance at age. Both estimators take into the spatial differences in age-length structures. These estimators are compared with the designed-based ALK estimator proposed by ICES for IBTS, which does not account for spatial differences in the age-length structure. As the proposed ALK estimator by ICES is a combination of age data over a large area, this can result in strongly biased estimates of numbers-at-age. An example of cod (\emph{Gadus morhua}) in ICES subareas IVa and IVb is used to illustrate spatial differences in the proportions of age-at-length, and estimates of uncertainty are presented using nonparametric bootstrapping. In general, the model-based ALK estimator provides a more accurate coverage probabilities compared with the other estimators.  


%Associated with the many parameters used in fish stock assessment is the uncertainty about their estimates, which cannot be ignored when formulating management policies \citep{walters1981effects, ludwig1981measurement}. This uncertainty can arise from many sources including natural variability, estimation procedures and lack of knowledge regarding the parameter \citep{ehrhardt1997role}. The North Sea International Bottom Trawl Surveys (NS-IBTS) data, coordinated by the ICES, provides information on seasonal distribution of stocks and estimates of abundance indices and catch in numbers of fish per age-class without an assessment of the accuracy of these estimates.  As pointed out by  \citet{ludwig1981measurement} estimates of parameters relating to stock size are of little value unless they are accompanied by estimates of measurement error variance.
% accounts for spatial variation in the  

%-two age-length 


% which is then used for stock assessments. No estimates of the accuracy of abundance indices 
\end{abstract}


%The North Sea International Bottom Trawl Survey (IBTS) was started by the International Centre for the  Exploration of the Sea (ICES) in 1990. Seven research vessels using standardized fishing methods participates in the survey. The survey with these vessels, which allows fishing also on rough ground provides information on seasonal distribution of stocks, abundance, hydrography and the environment  which is then used for stock assessments. Estimates of abundance indices based on age-length keys (ALK) are provided without any assessment of their accuracy.  We present a model-based ALK estimator, and a stratified design-based ALK estimator for estimating abundance at age. Both estimators take into the spatial differences in age-length structures. These estimators are compared with the designed-based ALK estimator proposed by ICES for IBTS, which does not account for spatial differences in the age-length structure. As the proposed ALK estimator by ICES is a combination of age data over a large area, this can result in strongly biased estimates of numbers-at-age. An example of cod (\emph{Gadus morhua}) in ICES subareas IVa and IVb is used to illustrate spatial differences in the proportions of age-at-length, and estimates of uncertainty are presented using nonparametric bootstrapping. In general, the model-based ALK estimator provides a more accurate coverage probabilities compared with the other estimators.  


\end{document}